\subsection{Ejemplo 1: Viaje en auto} Queremos minimizar la distancia recorrida sin quedarte sin nafta, 
teniendo un tanque con k cantidad de nafta que no puede ser recargado (por falta de presupuesto).
\paragraph{Modelado en un grafo} Los nodos son las ciudades (pueblos, estaciones, lugares tur�sticos o cualquier cruce de caminos).
Las aristas son las rutas que unen ciudades. El peso $w_1$ es la cantidad de nafta necesaria para 
recorrer ese tramo, y $w_2$ la distancia entre las ciudades. N�tese que no son dependientes $w_1$ y $w_2$ entre si; podemos necesitar mas
nafta para una distancia mas corta si, 
por ejemplo, tenemos que subir una cuesta o ir a una velocidad m�xima determinada.

\subsection{Ejemplo 2: S�per Minis-Vacaciones} Fuckensio lleva toda su vida ahorrando para irse de vacaciones a $Wolololandia$ y junto $K$ 
cantidad de dinero, del que destino $K'$ para el viaje de ida.
Como a Fuckensio le pagan por hora de trabajo independientemente del horario en que la realice, decidi� tomarse solo unos d�as de vacaciones, 
teniendo en cuenta hasta la ultima hora del 
viaje, por lo que necesita llegar
a la $Wolololandia$ lo mas r�pido posible, porque Fuckensio es pobre. Luego, usaremos este algoritmo para ayudarle a elegir el viaje �ptimo.
\paragraph{Modelado en un grafo} Los nodos son las ciudades donde empiezan y terminan los vuelos de la aerol�nea, que son representados por 
las aristas. El peso $w_1$ representa 
el costo del viaje y el peso $w_2$ representa el tiempo que dura el vuelo. Para simplificar el problema, asumimos que Fuckensio es una 
persona muy $Snob$ y como no quiere compartir 
el avi�n con nadie,alquilo un avi�n y pilotos privados por viaje.
Es decir, que puede partir inmediatamente luego de llegar al aeropuerto.

\subsection{Ejemplo 3: Pokemoon recargado.} Nuestra historia comienza en un momento critico del siglo 21. Un grupo anarquista llamado 
``Los Frikis Originales'', descontentos con 
la facilidad de los juegos actuales, realizo una protesta sobre los mismos. Dicho grupo determino que el mejor juego de la historia 
era el Pokemoon, por lo que 
comenz� a enviar cartas (y mails, y bombas) a las instalaciones de la compa��a GameBoi exigiendo nuevas versiones del juego Pokemoon
- Hoja Verde. Este desmadre de ``frikismo''
llego a tal nivel que los llamados ``Frikis Originales'' rastrearon a cada uno de los programadores del juego original y enviaron sus
reclamos a las casas de los mismos. 

Esto molesto a los programadores. Por lo tanto, anunciaron que crear�an un reto final, s�per dif�cil, que solo los jugadores veteranos
podr�an superar. Ahora bien, dichos programadores no
ten�an ganas de hacer 
un nuevo juego, por lo que se limitaron a aumentar la dificultad del original. Lograron esto, mediante el borrado de pedazos el c�digo.
Es decir, le sacaron jugabilidad al juego.

Ahora el nuevo reto consiste, obviamente, en ganar la Liga Pokemoon. Sin embargo, las reglas han cambiado. Ahora no existen los hospitales
Pokemoon, ni hay posibilidad de capturar o hacer evolucionar a 
los pokemoon, dado que ya no hay pokemoon salvajes (porque fueron todos capturaron por los frikis originales en la primera versi�n
del juego). Tampoco hay misiones extras. 
Lo que esto significa, es que debes
pasar por los gimnasios, uno tras otro, y llegar a la final, sin poder curar nunca a tus pokemoon. Y como que los programadores estaban 
realmente enfadados, decidieron aumentar la dificultad de la final, 
para que todos los jugadores llegaran con relativa facilidad hasta ella, pero ninguno pudiera pasarla.

Cuando el nuevo juego salio a la venta, nuestro protagonista Bill, un miembro de los ``Frikis Originales'', muy ilusionado descargo 
(ilegalmente) e instalo dicho juego a primera hora de su
ma�ana. Cuando se dispon�a a jugar, luego de finalizada la instalaci�n una hora mas tarde, apareci� su madre record�ndole que deb�a 
trabajar para independizarse, a sus 
cuarenta a�os, y que a las 17:00 hs pm tenia que presentarse en el McKonalls de junto para 
cumplir su horario laboral. Dado esto, a Bill le restaban 3 horas para terminar su juego, si quer�a llegar a tiempo.

Es por esto que Bill se encontraba en un dilema, pues quer�a terminar el juego y estimaba que la final pod�a costarle media hora de 
juego. Siendo tan friki como es, Bill puede estimar cuanto
tiempo le va a costar y llegar y
vencer en cada gimnasio con los pokemoon de que dispone, y cuan heridos van a terminar los mismos despu�s de cada combate. Como no sabe 
nada sobre la final, lo que Bill quiere lograr es minimizar el da�o total sufrido por el conjunto de sus pokemoon. 

Ahora bien, como Bill es un gran amigo nuestro, pues ha cuidado de nuestros pokemoon en su computadora durante mucho tiempo, 
decidimos ayudarlo modelando su problema con CACM. 

\paragraph{Modelado en un grafo} Los nodos son los gimnasios y las aristas representan los caminos entre ellos, si es posible llegar 
de uno a otro. El peso $w_1$ representa 
el tiempo que le lleva, al personaje de Bill, ir de un gimnasio a otro y vencer en dicho gimnasio. El peso $w_2$ representa el da�o 
que reciben los pokemoon en conjunto al finalizar cada combate.
La cota son las dos horas que le restan, pues hemos gastado media en escribir el grafo asociado al problema.

\paragraph{Problema adicional:} �A que hora comienza la ma�ana de Bill?
