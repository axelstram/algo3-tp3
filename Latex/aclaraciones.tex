\subsection{Sobre la complejidad }
 En todos los ejercicios trabajamos sobre el Modelo Uniforme. Por lo tanto suponemos en todos los casos que las asignaciones y comparaciones 
 de tipos b�sicos, definici�n de variables, creaci�n de iteradores y operaciones matem�ticas est�ndar tienen una complejidad constante.
 
\subsection{Sobre la implementaci�n }
La carpeta del tp contiene todos los archivos usados en la implementaci�n del problema, tambi�n contiene una subcarpeta llamada Tests, que contiene todo lo relativo al testeo del ejercicio. La carpeta de Tests contiene los generadores de instancias aleatorias, la experimentaci�n, y las comparaciones entre heur�sticas, que fueron usadas para crear los gr�ficos.
 
Tambi�n se adjuntan Makefile's para asegurar que el proceso de compilaci�n sea el mismo. 
 
 \subsection{Sobre la ejecuci�n }
 Para la ejecuci�n de los ejecutables, debe pasarse como par�metro un 1 o un 0 que indica si debe 
 tomar mediciones o no. 1 significa que si, 0 que no debe hacerlo, adem�s deben pasarse par�metros adicionales indicando que heuristica se desea aplicar(exacto, goloso, local y grasp). Finalmente, se debe usar el operador de redirecci�n <, seguido del nombre del archivo que contiene las instancias de test.
 La ejecuci�n de los generadores de instancias es directa, luego se pide ingresar los par�metros por consola, durante la ejecuci�n.  
 
  
 \subsection{Sobre la experimentaci�n }
A fin de reducir el $ruido$, ejecutamos 20 veces la resoluci�n de la misma instancia, y tomamos el tiempo final como
el promedio de los tiempos sumados.
\subsubsection{Medici�n de Tiempos}
Para realizar la medici�n de tiempos, utilizamos el tipo timespec.

